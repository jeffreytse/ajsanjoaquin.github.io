%!TEX TS-program = xelatex
%!TEX encoding = UTF-8 Unicode
% Awesome CV LaTeX Template
%
% This template has been downloaded from:
% https://github.com/posquit0/Awesome-CV
%
% Author:
% Claud D. Park <posquit0.bj@gmail.com>
% http://www.posquit0.com
%
% Template license:
% CC BY-SA 4.0 (https://creativecommons.org/licenses/by-sa/4.0/)
%


%%%%%%%%%%%%%%%%%%%%%%%%%%%%%%%%%%%%%%
%     Configuration
%%%%%%%%%%%%%%%%%%%%%%%%%%%%%%%%%%%%%%
%%% Themes: Awesome-CV
\documentclass[]{awesome-cv}
\usepackage{textcomp}
\hypersetup{%
  colorlinks=true,% hyperlinks will be coloured
  linkbordercolor=blue,% hyperlink border will be red
}
%%% Override a directory location for fonts(default: 'fonts/')
\fontdir[fonts/]

%%% Configure a directory location for sections
\newcommand*{\sectiondir}{resume/}

%%% Override color
% Awesome Colors: awesome-emerald, awesome-skyblue, awesome-red, awesome-pink, awesome-orange
%                 awesome-nephritis, awesome-concrete, awesome-darknight
%% Color for highlight
% Define your custom color if you don't like awesome colors
\colorlet{awesome}{awesome-red}
%\definecolor{awesome}{HTML}{CA63A8}
%% Colors for text
%\definecolor{darktext}{HTML}{414141}
%\definecolor{text}{HTML}{414141}
%\definecolor{graytext}{HTML}{414141}
%\definecolor{lighttext}{HTML}{414141}

%%% Override a separator for social informations in header(default: ' | ')
%\headersocialsep[\quad\textbar\quad]
\renewcommand*{\cventry}[5]{%
  \vspace{-2.0mm}
  \setlength\tabcolsep{0pt}
  \setlength{\extrarowheight}{0pt}
  \begin{tabular*}{\textwidth}{@{\extracolsep{\fill}} L{\textwidth - 4.5cm} R{4.5cm}}
    \ifempty{#2#3}
      {\ifempty{#1#4}{}{\entrypositionstyle{#1} & \entrydatestyle{#4} \\}}
      {\entrytitlestyle{#2} & \entrylocationstyle{#3} 
       \ifempty{#1#4}{}{\\\entrypositionstyle{#1} & \entrydatestyle{#4}}}
    \ifempty{#5}{}{\\\multicolumn{2}{L{\textwidth}}{\descriptionstyle{#5}}}
  \end{tabular*}%
  \par % <============================================================== missing in class
}

\usepackage{atbegshi,picture}
\AtBeginShipout{\AtBeginShipoutUpperLeft{%
  \put(\dimexpr\paperwidth-1cm\relax,-1.5cm){\makebox[0pt][r]{\framebox{Every colored text is a link.}}}%
}}

\begin{document}
%%%%%%%%%%%%%%%%%%%%%%%%%%%%%%%%%%%%%%
%     Profile
%%%%%%%%%%%%%%%%%%%%%%%%%%%%%%%%%%%%%%
\begin{center}
	\headerfirstnamestyle{\textbf{Ayrton}} \ \headerlastnamestyle{San Joaquin} \\
	\vspace{1mm}
	\headerpositionstyle{\faFlask \href{https://scholar.google.com/citations?user=e4UX388AAAAJ&hl=en}{Scientist \& ML Engineer, Trustworthy AI}} \\
	\vspace{1mm}
	{\faEnvelope\ ayrton@aya.yale.edu} | \iffalse {\faMobile\ +65 88147588} | \fi {\faMapMarker\ Singapore} | \faGithub\ \href{https://github.com/ajsanjoaquin}{ajsanjoaquin} | {\faLinkedinSquare\ \href{https://www.linkedin.com/in/ajsanjoaquin/}{ajsanjoaquin}}| {\faLegal\ \href{http://ajsanjoaquin.github.io/values}{Values}}
\end{center}
%%%%%%%%%%%%%%%%%%%%%%%%%%%%%%%%%%%%%%
%     Education
%%%%%%%%%%%%%%%%%%%%%%%%%%%%%%%%%%%%%%
%\vspace{1mm}
\cvsection{Education}
\begin{cventries}
	\cventry
	{BSc. (Honors) in Data Science, Minor in Philosophy. \textbf{(Scholar, With High Distinction)}}
	{Yale-NUS College}
	{Singapore}
	{August 2018 - May 2023}
	{Semester Abroad at the University of Copenhagen, Denmark}
\end{cventries}
\vspace{-3mm}
%%%%%%%%%%%%%%%%%%%%%%%%%%%%%%%%%%%%%%
%     Experience
%%%%%%%%%%%%%%%%%%%%%%%%%%%%%%%%%%%%%%
\cvsection{Experience}
\begin{cventries}
	\cventry
	{AI Scientist, \href{https://descartes.cnrsatcreate.cnrs.fr/}{DesCartes program} (https://descartes.cnrsatcreate.cnrs.fr/)}
	{French National Centre for Scientific Research (CNRS)@CREATE}
	{Singapore}
	{September 2023 - Present}
	{\begin{cvitems}
		\item {Leading a study on efficient fine-tuning of open-source Large Language Models (LLMs) for instruction-following.}
		\item {Fine-tuned dozens of SOTA LLMs (Llama-\{3,2\}, Mistral, Mixtral, Phi-2, Gemma, TinyLlama) on Ultrachat dataset (200k) and Alpaca (52.2k)}
		\item {Implemented automated training scripts in distributed settings ranging from 1-2 node clusters provided by the \href{https://www.nscc.sg/}{Singapore National Supercomputer}.}
	\end{cvitems}}
	\vspace{-2mm}
	\cventry
	{Scholar}
	{Machine Learning Safety Scholars Program, Center for AI Safety}
	{Palo Alto, United States}
	{June 2022 - August 2022}
	{\begin{cvitems}
		\item {Performed prompt-injection attacks against LLMs (GPT-3, LaMDA, T5) via API access.}
		\item{Implemented various strategies in \textbf{robustness} (PGD, adversarial training), \textbf{anomaly detection} (AUROC, ViM), \textbf{calibration} (RSME, Brier scores), and \textbf{trojan attacks} (data poisoning).}
	\end{cvitems}}
	\vspace{-2mm}
	\cventry
	{Research Engineer}
	{Data Privacy and Trustworthy Machine Learning Lab, NUS}
	{Singapore}
	{May 2021 - March 2022}
	{\begin{cvitems}
		\item {Collaborated with \textbf{Google DeepMind} on privacy and adversarial machine learning research for my bachelor's thesis in a team across 4 time zones. \textbf{Published in a top security conference (ACM CCS) as the youngest and only undergraduate co-author.}}
	\end{cvitems}}
	\vspace{-2mm}
	\cventry
	{Deep Learning Engineer}
	{Arterys (Freelance)}
	{San Francisco, United States}
	{March 2020 - June 2020}
	{\begin{cvitems}
		\item {Created a COVID-19 Pneumonia classifier \textbf{4 days after pandemic declaration in collaboration with A.I. Singapore} based on a ResNet-38.}
		\item {Collaborated with Arterys to \href{https://www.linkedin.com/feed/update/urn:li:activity:6676702441607716864/}{deploy the model in their platform} 
		for use by American hospitals and researchers. Model engineer in a team of 4 across 3 time zones.}
	\end{cvitems}}
\end{cventries}

\vspace{-5mm}
\cvsection{AI Engineering Projects}
\begin{cventries}
	\cventry
	{}
	{DesCartes Program Semantic Search Engine}
	{LLMs}
	{}
	{• Deployed a system internally to perform document retrieval via vector-based semantic search. Accepts and returns multilingual queries in English and French. The system is composed of a Llama-3 (8B) model for chat and bge-small model for embeddings. Orchestrated via LlamaIndex.}
		
	\vspace{-1mm}
	\cventry
	{}
	{\href{https://github.com/ajsanjoaquin/meta-aria}{Meta Project Aria Workshop 2023}}
	{LLMs, Contextual AI}
	{}
	{• \textbf{Invited by Meta Reality Labs} to design a use-case for \href{https://about.meta.com/en/realitylabs/projectaria/}{Project Aria}. Created an LLM assistant that uses social media data and real-time visual context from smart glasses. \href{https://drive.google.com/file/d/1ymkkCU31EP57DDXJpdB9NIuVrsMtZB9t/view?usp=drive_link}{Video demo.}}

	\vspace{-1mm}
	\cventry
	{}
	{Document Summarization}
	{LLMs}
	{}
	{• Fine-tuned GPT-J (6B) on 100k arxiv pre-prints to summarize research papers used by my lab.}
	
	\vspace{-1mm}
	\cventry
	{}
	{\href{https://github.com/ajsanjoaquin/mPerturb}{Eplaining Neural Networks with Meaningful Perturbations}}
	{Explainable AI}
	{}
	{• For explaining an image classifier's prediction, I implemented the algorithm described in \textit{Explanations of Black Boxes by Meaningful Perturbation (Fong, et. al., 2018)}.}
	
	\vspace{-1mm}
	\cventry
	{}
	{\href{hhttps://github.com/ajsanjoaquin/Shapley_Valuation}{Equitable Valuation of Data Using Shapley Values}}
	{Explainable AI}
	{}
	{• I did a Pytorch implementation of computing Shapley values via Truncated Monte Carlo sampling from \textit{What is your data worth? Equitable Valuation of Data (Ghorbani and Zou, 2019)}.}
	
	\vspace{-1mm}
	\cventry
	{}
	{\href{https://github.com/ajsanjoaquin/COVID-19-Scanner}{COVID-19 Medical Triage Model}}
	{Computer Vision}
	{}
	{• I developed a model meant to help triage patients (prioritize certain patients for testing, quarantine, and medical attention) that require diagnosis for COVID-19. Trained on 26k x-ray images.}
	
	\vspace{-1mm}
	\cventry
	{}
	{\href{https://github.com/pulls?q=is\%3Apr+author\%3Aajsanjoaquin+archived\%3Afalse+is\%3Aclosed}{Open-Source AI}}
	{DevOps}
	{}
	{• Added new features for major machine learning projects including \textbf{Pytorch, HuggingFace Transformers, and YOLOv4} (object detection model).}
\end{cventries}

\cvsection{Skills}
\vspace{-3mm}
\begin{cventries}
	\cventry
	{}
	{\def\arraystretch{1.15}{\begin{tabular}{ l l }
		Machine Learning:  & {\skill{ Pytorch, Tensorflow, LlamaIndex, JAX, HuggingFace, Langchain, NLTK, Spacy}} \\
		Data:  & {\skill{Pandas, PySpark, Querying (SQL, MongoDB), Vector Database (Qdrant, Pinecone)}} \\
		MLOps: & {\skill{Linux, Databricks, GCP, AzureML, Snowflake, Docker, Flask, Continuous Integration, Kubernetes}} \\
	\end{tabular}}}
	{}
	{}
	{}
\end{cventries}

\end{document}